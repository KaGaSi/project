\subsection[VtfReadTimestep]{Read single timestep from \vtf coordinate
file}\label{ssec:VtfReadTimestep}

This function reads a single timestep (its preamble and the coordinate block)
from an input \vcf file, identifying what beads are present via
\ttb{Bead[].InTimestep} flag.

\CallInit{void VtfReadTimestep}
\begin{longtable}{L{0.36\textwidth}Mp{0.49\textwidth}}
  \toprule
  parameter          & in/out & explanation \\
  \midrule
  \ttb{(FILE) vcf}                   & in  & pointer to open \vcf file\\
  \ttb{(char) vcf_file[]}            & in  & name of the input \vcf file\\
  \ttb{(BOX) Box}                    & out & box dimensions (section
                                             \ref{ssec:Box})\\
  \ttb{(COUNTS) Counts}              & out & basic system information
                                             (section \ref{ssec:Counts})\\
  \ttb{(BEADTYPE) BeadType[]}        & out & information about bead types
                                             (section \ref{ssec:BeadType})\\
  \ttb{(BEAD) Bead[]}                & out & information about individual beads
                                             (section \ref{ssec:Bead})\\
  \ttb{(int) Index[]}                & out & array connecting internal and  \vtf
                                             bead indices\\
  \ttb{(MOLECULETYPE) MoleculeType[]}& in \& out & information about molecule types
                                             (section \ref{ssec:MoleculeType})\\
  \ttb{(MOLECULE) Molecule[]}        & out & information about individual
                                             molecules (section
                                             \ref{ssec:Molecule})\\
  \ttb{(int) step_count}             & in \& out & counter of read timesteps\\
  \bottomrule
\end{longtable}
\begin{enumerate}
  \item read \ttb{struct_file} line by line
    \begin{itemize}
      \item save all information from \tt{atom} and \tt{bond} lines
        \begin{itemize}
          \item only the first \tt{atom default} line is saved
        \end{itemize}
      \item count
        \begin{itemize}
          \item beads (i.e., identify the highest index in \tt{atom <id>}
            lines), saving into \ttb{Counts.BeadsInVsf}
          \item molecules, saving into \ttb{Counts.Molecules}
          \item unique bead and molecule names (and save those names)
          \item \tt{atom} and \tt{bond} lines (\TODO necessary to mention here?)
        \end{itemize}
      \item stop reading when
        \begin{itemize}
          \item end of file or \tt{timestep} line encountered
          \item unrecognised line or coordinate line encounted and exit program
            with error
        \end{itemize}
        \begin{itemize}
          \item end of file or \tt{timestep} line encountered
          \item unrecognised line or coordinate line encounted and exit program
            with error
        \end{itemize}
    \end{itemize}
  \item save number of unbonded and bonded beads into \ttb{Counts.Unbonded} and
    \ttb{Counts.Bonded}
  \item identify bead types
    \begin{itemize}
      \item save the number of types to \ttb{Counts.TypesOfBeads}
      \item fill a temporary \nameref{ssec:BeadType} array
    \end{itemize}
    \begin{enumerate}
      \item if \ttb{detailed} is \tt{true}, identify the types based on names,
        mass, charge, and radius
        \begin{enumerate}
          \item create a new bead type for every \tt{atom} line that has a
            unique combination of name, mass, charge, and radius
          \item merge some of the bead types, specifically:
            \begin{itemize}[label=$-$]
              \item if a keyword is missing in one \tt{atom} line but present
                in another does not count as a different type; e.g. beads\\
                \tt{atom 0 n x m 1 q 1}\\
                \tt{atom 1 n x m 1}\\
                are of the same type (with charge $+1$)
              \item however, there can be ambiguities, so e.g., beads\\
                \tt{atom 0 n x m 1 q 1}\\
                \tt{atom 1 n x m 1}\\
                \tt{atom 2 n x m 1 q 0}\\
                remain as three distinct types (with charges 0, \tt{undefined},
                and $+1$)
              \item but only some \tt{atom} lines are ambiguous; e.g., beads\\
                \tt{atom 0 n x m 1 q 1}\\
                \tt{atom 1 n x m 1}\\
                \tt{atom 2 n x m 1 q 0}\\
                \tt{atom 3 n x q 0}\\
                are still of three types with the same charges as above and
                with mass 1 (there is no ambiguity as \tt{atom 3} is the same as
                \tt{atom 2} except for the undefined mass)
              \item note that sometimes the charge, mass, and/or radius can
                remain undefined even though there is exactly one well defined
                value; e.g., in case of \tt{atom} lines\\
                \tt{atom 0 n x m 1 q 1}\\
                \tt{atom 1 n x m 1}\\
                \tt{atom 2 n x m 1 q 0}\\
                \tt{atom 3 n x q 0}\\
                \tt{atom 4 n x m 1 q 0 r 1}\\
                beads sharing their type with \tt{atom 4} will have well
                defined radius of 1 (i.e., \tt{atom 2}, \tt{3}, and \tt{4}),
                while \tt{atom 0} and \tt{1} will have radius undefined as they
                have different charge to \tt{atom 4}; three bead types are
                recognized here:
                  \begin{enumerate}
                    \item type with $m=1$, $q=1$, and radius \tt{undefined}
                      (\tt{atom 0})
                    \item type with $m=1$ and charge and radius \tt{undefined}
                      (\tt{atom 1})
                    \item type with $m=1$, $q=0$, and $r=1$ (\tt{atom 2},
                      \tt{3}, and \tt{4})
                  \end{enumerate}
            \end{itemize}
        \end{enumerate}
      \item if \ttb{detailed} is \tt{false}, identify the types by name only
        \begin{itemize}
          \item mass, charge, and radius for each bead type is taken from the
            first \tt{atom} line of the given name where the respective
            parameter is defined
        \end{itemize}
    \end{enumerate}
  \item fill a temporary \nameref{ssec:Bead} array and a temporary associated
    index (i.e., \ttb{Index}) array
    \begin{itemize}
      \item put unbonded beads first and the bonded beads after them
      \item if \ttb{detailed} is \tt{false}, only bead name is checked to
        determine its type, otherwise its name, mass, charge, and radius are all
        checked
    \end{itemize}
  \item if \ttb{detailed} is \tt{true}, rename bead types with the same name
    \begin{itemize}
      \item when several bead types share a name, the name remains unchanged for
        the first one, but \tt{_\#} is added to all subsequent ones (\tt{\#}
        goes from 1 to $N$, where $N$ is the number of bead types with the same
        name)
      \item if the bead name would become too long (i.e., over 16 characters),
        it is shortened before the \tt{_\#} is added
    \end{itemize}
  \item identify molecules and molecule types
    \begin{itemize}
      \item save the number of types to \ttb{Counts.TypesOfMolecules}
      \item fill temporary \nameref{ssec:MoleculeType} and
        \nameref{ssec:Molecule} arrays
      \item molecules must share all information to be of the same type
        \begin{itemize}
          \item molecule name
          \item numbers of beads and bonds
          \item connectivity (i.e., the bonds must be identical)
          \item order of bead types (i.e., the order of the beads' indices in
            \ttb{struct_file})
        \end{itemize}
    \end{itemize}
  \item copy data from temporary arrays to \ttb{Bead}, \ttb{BeadType},
    \ttb{Index}, \ttb{Molecule}, and \ttb{MoleculeType}
\end{enumerate}

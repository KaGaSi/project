\section{GenLayers} \label{sec:GenLayers}

This utility generates two monolayers composed of molecules specified in a
\texttt{FIELD}-like file. Note that only one type of molecule is used; if
more complex systems are required, several different layers can be
generated separately (allowing for better control of the final complex
system) and joined using the \texttt{AddToSystem} utility
(Section~\ref{sec:AddToSystem}). The two layers are mirror images of each
other, that is, molecules in both layers are grown from box edge to box
middle. The layers are placed in $z$ direction (or on $xy$ planes of
the simulation box).

The first beads of the molecules are arranged on a square lattice defined
either by given spacing in $x$ and $y$ directions (\texttt{-s} option) or
by the number of molecules per layer (\texttt{-nm} option); the default is
spacing of 1 in both $x$ and $y$ directions. The rest of the beads of each
molecule are placed based on the coordinates in the \texttt{FIELD}-like
file. By default, \texttt{GenLayers} places the two mirror layers at the
edges of a simulation box; using \texttt{-g} option, a gap from the box
edge can be introduced. Therefore, this utility can generate, for example,
polymer brushes at box edges or a double layer (such as a biological
membrane) in the middle of the box.

By default, the total number of beads in the generated system is equal to
three times the box volume, that is, the typical number of beads in
dissipative particle dynamics simulations. Beads that are not in the
molecules, are put at the beginning of the output files (i.e., they have
lower indices) with coordinates of the box centre, and name \texttt{None}.
The idea is that once the layers are generated, \texttt{AddToSystem}
utility (Section~\ref{sec:AddToSystem}) can be used to exchange these
excess beads for different species (using its \texttt{--switch} option).
The \texttt{-n} option changes the total number of beads. If the number is
lower than the total number of beads needed to construct the two layers of
molecules, their amount is adjusted to exactly that number.

The input \texttt{FIELD}-like file must contain \texttt{species} and
\texttt{molecule} sections (although, only the first molecule is considered
and \texttt{nummols} is disregarded), but the \texttt{interaction} section
is ignored (see \texttt{DL\_MESO} manual for details on the \texttt{FIELD}
file or the \texttt{Example/GenLayers} directory for a simple example). The
first line of the \texttt{FIELD}-like file, must start with box dimensions,
i.e., with three numbers.

The utility generates \texttt{vsf} structure and \texttt{vcf} coordinate
files. An example of the usage is shown in \texttt{Example/GenLayers}
directory.

Usage (\texttt{GenLayers} does not use standard options):

\vspace{1em}
\noindent
\texttt{GenLayers <out.vsf> <out.vcf> <options>}

\noindent
\begin{longtable}{p{0.15\textwidth}p{0.794\textwidth}}
  \toprule
  \multicolumn{2}{l}{Mandatory arguments} \\
  \midrule
  \texttt{<out.vsf>} & output \texttt{vsf} structure file \\
  \texttt{<out.vcf>} & output \texttt{vcf} coordinate file \\
  \toprule
  \multicolumn{2}{l}{Options} \\
  \midrule
  \texttt{-s <x> <y>} & spacing of molecules in $x$ and $y$ directions
    (default: 1 1) \\
  \texttt{-n <int>} & total number of beads (default: three times the box
    volume)\\
  \texttt{-nm <int>} & number of molecules in each layer (rewrites
    \texttt{-s} option) \\
  \texttt{-g <float>} & gap between box edges and the molecules (default: 0)\\
  \texttt{-f <name>} & \texttt{FIELD}-like file (default: \texttt{FIELD}) \\
  \texttt{-v}        & verbose output that provides information about all
    bead and molecule types \\
  \texttt{-h}        & print this help and exit \\
  \bottomrule
\end{longtable}

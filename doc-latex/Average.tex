\section{Average} \label{sec:Average}

This utility calculates one of three types of averages based on a used option
from specified column(s) of data in the input text file (all \tt{\#}-starting
lines and blank lines are ignored). It either produces an overall average with
statistical error and an autocorrelation time estimate (\tt{-tau} option),
block averages (\tt{-b} option), or moving averages (\tt{-m} option). While the
\tt{-tau} option appends a single line to the output file, either of the
\tt{-b} or \tt{-m} options creates a new output file with somewhat smoother
data.

The first and last lines used for the average calculation can be controlled
using the standard \tt{-st} and \tt{-e} options.

\subsection{Estimate autocorrelation time via -tau option}

The average value of an observable $\mathcal{O}$ is a simple arithmetic
mean,
\begin{equation} \label{eq:Average} %{{{
  \langle\mathcal{O}\rangle = \frac{1}{N} \sum^N_{i=1} \mathcal{O}_i,
\end{equation} %}}}
where $N$ is the number of measurements and the subscript $i$ denotes
individual measurements. If the measurements are independent (i.e.,
uncorrelated), the statistical error, $\epsilon$, is given by:
\begin{equation} \label{eq:IndependentError} %{{{
  \epsilon^2 =
    \frac{\sigma^2_{\mathcal{O}_i}}{N},
\end{equation} %}}}
where $\sigma^2_{\mathcal{O}_i}$
is the variance of the individual
measurements,
\begin{equation} %{{{
  \sigma^2_{\mathcal{O}_i} = \frac{1}{N-1} \sum^N_{i=1} (\mathcal{O}_i -
  \langle\mathcal{O}\rangle)^2.
\end{equation} %}}}

For correlated data, the autocorrelation time,
$\tau$, representing the number of steps between two uncorrelated
measurements must be determined. Every
$\tau$-th measurement is uncorrelated, so the
equation~\eqref{eq:IndependentError} can then be used to estimate the
error.

A commonly used method to estimate $\tau$
is the binning (or block) method. In this method, the correlated data are
divided into $N_{\mr{B}}$ non-overlapping blocks of size $k$ ($N=k
N_{\mr{B}}$) with per-block averages, $\mathcal{O}_{\mr{B},n}$,
defined as:
\begin{equation} \label{eq:BlockAverage} %{{{
  \mathcal{O}_{\mr{B},n} = \frac{1}{k}
    \sum^{k n}_{\substack{i=1+\\(n-1)k}} \mathcal{O}_i.
\end{equation} %}}}
If $k\gg\tau$, the blocks are assumed to be uncorrelated and
equation~\eqref{eq:IndependentError} can be used:
\begin{equation} \label{eq:Error} %{{{
  \epsilon^2 =
  \frac{\sigma^2_{\mr{B}} }{N_{\mr{B}} } = \frac{1}{N_{\mr{B}} (N_{\mr{B}}
  -1)} \sum^{N_{\mr{B}} }_{n=1} (\mathcal{O}_{\mr{B},n} -
  \overline{\mathcal{O}})^2.
\end{equation} %}}}
An estimate of the autocorrelation time can be obtained using the following
formula:
\begin{equation} \label{eq:tau} %{{{
  \tau_{\mathcal{O}} = \frac{k \sigma^2_{\mr{B}} }{2
  \sigma^2_{\mathcal{O}_i}}.
\end{equation} %}}}

The number of blocks, $N_{\mr{B}}$, is supplied as an argument of the \tt{-tau}
option and Average then appends a single line to the \tt{<output>} file; the
line starts with $N_{\mr{B}}$ and continues with three values
($\langle\mathcal{O}\rangle$, $\epsilon$, and $\tau_{\mathcal{O}}$) per every
data column specified in the Average command.

A way to quickly get a $\tau$ estimate is to use a wide range of $N_{\mr{B}}$
values and plot $\tau_{\mathcal{O}}$ from equation~\eqref{eq:tau} as a function
of $N_{\mr{B}}$. Because the number of data points in one block should be
significantly larger than the autocorrelation time (e.g., ten times larger),
plotting $f(x)=N/(10x)$ will produce a monotonously decreasing curve that
intersects the $\tau_{\mathcal{O}}$ vs. $N_{\mr{B}}$ curve. A value of
$\tau_{\mathcal{O}}$ near the intersection (but to the left, where the
decreasing curve is above $\tau_{\mathcal{O}}$ vs. $N_{\mr{B}}$ curve) can be
considered a safe estimate of $\tau$.

\subsection{Block averages via -b option}

Besides estimating the autocorrelation time, the per-block averages can be
themselves plotted to get (probably) smoother dataset. Using the \tt{-b} option,
the number of data points per block to average, $k$, is supplied, and the utility
prints the per-block averages from equation~\eqref{eq:BlockAverage} to the
output file.

This way of averaging could be useful for example with density data produced by
DensityBox or related utilities; if the bin width supplied to the DensityBox was
too small, it is (possibly much) faster to block-average the densities rather
than rerun DensityBox. For this case, specify the first column (distance) along
with any density columns from the density file.

\subsection{Moving averages via -m option}

More common way to smoothing noisy data is to use the moving (or rolling or
running) average (or moving mean or rolling mean); this common method does have
many names.

Similarly to the block-average, the first element of the moving average is a
simple mean of $k$ values. Unlike with block-average, however, the $k$ values
for the next element are obtained by ignoring only one value and taking the next
$k$ values; i.e., $k-1$ values from the previous element of the moving average
are always reused as opposed to the block-average where the blocks of data are
not overlapping.

The moving average elements, $\mathcal{O}_{\mr{M},n}$, are defined as
\begin{equation} \label{eq:MovingAverage}
  \mathcal{O}_{\mr{M},n} = \frac{1}{k}
    \sum^{n+k-1}_{n} \mathcal{O}_i.
\end{equation}

\vspace{1em}
\noindent
Usage: \tt{Average <input> <output> <column(s)>  [options]}
\noindent
\begin{longtable}{p{0.15\textwidth}p{0.794\textwidth}}
  \toprule
  \multicolumn{2}{l}{Mandatory arguments} \\
  \midrule
  \tt{<input>}     & input text file\\
  \tt{<output>}    & output text file\\
  \tt{<column(s)>} & at least one column number from \tt{<input>}\\
  \midrule
  \midrule
  \multicolumn{2}{l}{Options}\\
  \midrule
  \tt{-tau <int>} & $\tau$ estimation where \tt{<int>} represents the number of
                    blocks $N_{\mr{B}}$ \\
  \tt{-b <int>}   & block-average printing mode where \tt{<int>} represents
                    the number of data points per one block, $k$
                    (equation~\eqref{eq:BlockAverage})\\
  \tt{-m <int>}   & moving average mode where \tt{<int>} represents the number
                    of data points per one moving block, $k$
                    (equation~\eqref{eq:MovingAverage})\\
  \midrule
  \multicolumn{2}{l}{Other options (see the beginning of 
                     Chapter~\ref{chap:Utils})}\\
  \midrule
  \multicolumn{2}{l}{\tt{-st},
                     \tt{-e},
                     \tt{--verbose},
                     \tt{--silent},
                     \tt{--help},
                     \tt{--version}}\\
  \bottomrule
\end{longtable}

\chapter{Format of input/output files}\label{chap:input}

This section describes several file types used by many of the
AnalysisTools utilities. Output files for the utilities themselves are
described in their respective sections in \cref{chap:Utils}.

Input files are divided into two categories: structure and coordinate files. The
structure files contain information about the system composition, that is,
number of beads and molecules, bead properties (name, mass, charge, etc.),
molecule properties (name, number and types of beads, connectivity, etc.). The
coordinate files contain individual timesteps with beads' coordinates (and,
possibly, velocities, forces, etc.). While most files can be used as both a
structure and a coordinate file, using a separate structure file may be
essential. For example, an \tt{xyz} file (see below) may be used as both, but
because it only provides bead names, using a separate structure file would
provide additional information (bead and/or molecule properties).

For working with aggregates, AnalysisTools' \tt{agg} file format (see below)
describes their compositions in terms of molecule numbers taken from the
structure file used to generate it. Any of the \tt{Aggregates}
(\cref{sec:Aggregates}) utilities can create the \tt{agg} file.

\section{Structure files}\label{sec:StructFile}
As a structure file, any of the following formats can be used; browse the
\tt{Examples} directory for example files. Most utilities can load a specific
structure file using the \tt{-i <file>} option; if the option is not used, it is
usually assumed the structure file is the same as the coordinate file.

Importantly, many of the AnalysisTools utilities work (either fundamentally or
optionally) with types of beads and/or molecules. In these cases, names are used
specify the bead/molecule types. While many input files support naming beads and
molecules, not all do (in some formats, these are optional); in such cases,
AnalysisTools names the bead types as \tt{b} (and \tt{b1}, \tt{b2}, etc., if
more than one bead type is present) and molecule types as \tt{m} (and \tt{m1},
\tt{m2}, etc., in case of multiple molecule types). If unsure, using the Info
utility (\cref{sec:Info}), one can see the names of beads and molecules for the
specific structure file.

\tt{Info} utility can also convert these formats between each other via the
\tt{-o <file>} option.

\subsection{VTF format}\label{ssec:StructVtf} %{{{

See \href{https://github.com/olenz/vtfplugin/wiki/VTF-format}{here} for the
complete description of \vtsf files (a \vtf file contains a structure part
followed by a coordinate part, while a \vsf file contains only the structure
part). In this format, every bead is defined on its own line (except for the
\tt{atom default} bead type) and all bonds all listed.

Not all the keywords listed on the web page for the \enquote{Atom lines} are
recognized; keyword \tt{n[ame]} is mandatory and optional keywords are
\tt{m[ass]}, \tt{charge|q}, \tt{r[adius]}, \tt{resid}, and \tt{res[name]}. For
the \tt{<aid-specifier>}, only a single bead index number or the \tt{default}
keyword can be used. Other keywords are ignored.

AnalysisTools groups beads (and molecules) into bead (and molecule) types.
Generally, only name defines types of beads, i.e., should two beads share the
name, they will be of the same type; their mass, charge, and radius will each
equal to that for the bead type’s topmost atom line containing the corresponding
keyword. If the keyword is missing from all atom lines, that characteristic is
undefined. In the case of the \tt{Info} utility (see \cref{sec:Info} for
details), the \tt{--detailed} command line option triggers the detailed
recognition, where beads that share the name but differ in mass, charge, and/or
radius will be of different type (with a number appended to the original name).

Different molecule types, on the other hand, are always distinguished based on
all their characteristics, i.e., name, connectivity, and order of bead types
(order of its \vsf indices). Molecule types may also be affected by the
\tt{--detailed} option because of the bead type definition.

For bond lines, only one bond per line is permitted (i.e., \tt{bond
<index1>:<index2>} format).

Periodic boundary condition can be included as \tt{pbc <x> <y> <z>} for a cuboid
box or \tt{pbc <a> <b> <c> <alpha> <beta> <gamma>} for a general rhomboid
box. %}}}

\subsection{FIELD format}\label{ssec:StructField} %{{{

A format used by the
\href{https://www.scd.stfc.ac.uk/Pages/DL_MESO.aspx}{DL_MESO_DPD simulation
package} concisely defines species of beads and molecules, providing number of
beads/molecules of given species as well as their properties.

The \tt{species} section is the same as in the \tt{DL_MESO} \field file, but the
\tt{molecules} section differs slightly. Besides the \tt{beads} part, every
molecule can have \tt{bonds}, \tt{angles}, and \tt{dihedrals}, but even though
up to three parameters for potential parameters are read, AnalysisTools does not
recognize the potential keywords (\tt{harm}, \tt{cos}, etc.). Moreover, these
potential parameters are only used for generating new \tt{LAMMPS} or
\tt{DL_MESO} data files via, e.g., the \tt{Info} utility (\cref{sec:Info}). The
\tt{molecules} section can also contain one extra part: \tt{impropers} for
improper angles which has the same format as the \tt{dihedrals} section (this
was added because the \href{https://lammps.sandia.gov/}{LAMMPS molecular
dynamics simulator} and other software distinguish between the two types).

Note that the \field file does not have to contain the \tt{Interactions}
section; this part is ignored.

However, using this file in conjunction with coordinate files is not
recommended. Because \field contains only numbers of beads of given types, it is
not possible to ensure that bead indices in the coordinate file will line up
with the correct bead types in the \field file. AnalysisTools assigns the bead
indices on the \enquote{first read, first labelled} basis, that is, the unbonded
beads are first (in the order of the lines in the \tt{species} section) followed
by beads from the molecules. Should one want to check the bead index
assignments, the \tt{Info} utility can be used to generate, e.g., a \vsf file
from the \field file.

Nevertheless, the file can be used to supply extra information for bead and
molecule types for the \tt{Info} utility's \tt{-i} option as these do not
depend on individual beads but rather on bead and molecule type names (see
\cref{sec:Info} for details). Moreover, \field is used as an input file to add
extra species into an existing system or to create a new system from scratch
via, e.g., \fullref{sec:AddToSystem} or \TODO{ref GenSystem, GenLayers, or some
such}.

To be recognized by AnalysisTools utilities, the file must either be called
\field or have the \tt{.FIELD} extension.
%}}}

\subsection{XYZ format}\label{ssec:StructXyz} %{{{

This is the simplest and probably best known format, see, e.g.,
\href{https://openbabel.org/docs/dev/FileFormats/XYZ_cartesian_coordinates_format.html}{here}
for the description. An \xyz file is essentially a file with timesteps (bead
coordinates), and the only structure information are bead names. Note that the
second line of each timestep (a comment line) is ignored.

When used as a structure file, the first timestep is used to define the system
composition. Therefore while the timesteps can each have a different number of
beads, their amount cannot exceed the number in the first step.

% The comment line at the beginning of each timestep can contain periodic boundary
% conditions as \tt{<x> <y> <z>} (cuboid box) or \tt{<a> <b> <c> <alpha> <beta>
% <gamma>} (general rhomboid box); it is similar to the \tt{pbc} keyword in \vtf
% file. To use this box definition, use \tt{-pbc <n>} option where \tt{<n>} is the
% position of the first number on the comment line (e.g., \tt{1} if the comment
% line starts with the box size).
%}}}

\subsection{LAMMPS data format}\label{ssec:StructLdata} %{{{

This rather complicated format comes from the
\href{https://lammps.sandia.gov/}{LAMMPS molecular dynamics simulator} and lists
the numbers of beads, bonds, angles, etc., as well as individual beads, bonds,
angles, etc., and information about potentials for bonds, angles, etc. See
\href{https://docs.lammps.org/2001/data_format.html}{here} for description of
the format and the \tt{Examples} directory for example files.

AnalysisTools reads the header information as well as \tt{Masses}, \tt{Atoms},
\tt{Velocities}, \tt{Bonds}, \tt{Angles}, \tt{Dihedrals}, \tt{Impropers}. It
also reads \tt{Bond Coeffs}, \tt{Angle Coeffs}, \tt{Dihedral Coeffs}, and
\tt{Improper Coeffs} sections.

The \tt{Atoms} section may have \tt{atom_style full}, \tt{bond}, \tt{angle},
\tt{atomic}, \tt{molecular}, or \tt{charge} format (specified as a trailing
comment on the \tt{Atoms} line as required by the \tt{LAMMPS} software).

For the \tt{Bond Coeffs} and \tt{Angle Coeffs} sections, harmonic potential is
assumed, and the first value of each bond/angle type is multiplied by 2 because
LAMMPS uses harmonic spring strength of $k/2$ (as opposed to, e.g., DL_MESO
which uses $k$). For the \tt{Dihedral Coeffs} and \tt{Improper Coeffs} sections,
AnalysisTools reads at most three numbers without assuming any specific
potential. The \tt{Coeffs} sections are used only for generating new structure
files via, e.g., the \fullref{sec:Info} utility; note that to run \tt{LAMMPS}
simulations, these section may have to be manually corrected to conform to the
specific potential type.

To be recognized by AnalysisTools utilities, the file must have an extension
\tt{.data}. %}}}

\subsection{LAMMPS lammpstrj format}\label{ssec:StructLtrj} %{{{

This file format comes from the \tt{LAMMPS}
\href{https://docs.lammps.org/dump.html}{dump style custom} command.

Of the possible \tt{LAMMPS} attributes, AnalysisTools recognizes only \tt{id}
(bead index -- mandatory), \tt{element} (bead name), \tt{x y z} (bead
coordinates), \tt{vx vy vz} (bead velocities), and \tt{fx fy fz} (bead forces).
Other attributes are ignored. Note that if \tt{element} is missing, all beads
are considered of the same type called \tt{b}.

When used as a structure file, the first timestep is used to define the system
composition. Therefore while the timesteps can each have a different number of
beads, their amount cannot exceed the number in the first step.

To be recognized by AnalysisTools utilities, the file must have an extension
\tt{.lammpstrj}. %}}}

\section{Coordinate files}\label{sec:CoorFile}
As a coordinate file, any of the following formats (described above) can be
used:
\begin{itemize}
  \item \vtcf format (\vtf file contains both structure and coordinate sections
    while \vcf file contains only the coordinate section); both indexed and
    ordered timesteps can be used, but the AnalysisTools utilities always output
    files with ordered timesteps
  \item \xyz format where timesteps can have a different number of beads
  \item LAMMPS \data format which by definition can only contain one timestep
  \item LAMMPS \ltrj format where timesteps can have a different number of beads
\end{itemize}

\section{Aggregate file} \label{sec:AggFile} %{{{

An \tt{agg} is generated using any of the \tt{Aggregates} utilities. The file
contains information about the number of aggregates in each timestep and which
molecules
% and monomeric (i.e., unbonded) beads 
belong to which aggregate. It serves as an additional input file for utilities
that calculate aggregate properties; \tt{agg} file is, therefore, linked to the
structure and coordinate file(s) used to generate it.

The \tt{agg} file is a simple text file. The first two lines are just comments
(the second one should contain the command used to generate the file as parts of
the command may be used by subsequent aggregate analysis). Starting at the third
line, the data for individual timesteps are shown. It follows these rules:

\begin{itemize}[topsep=0pt,itemsep=0pt]
  \item each timestep starts with \tt{Step: <int>}
  \item the second line is the number of aggregates in the given timestep
  \item for each aggregate, there is a single line in the format \tt{<size> :
    <id1> <id2> ...}; the \tt{<id\#>} are molecular indices from the
    input structure file
  \item no blank or comment lines are allowed
  \item not all molecules present in the coordinate/structure file(s) used to
    generate this file must be present in every timestep
  \item the keyword \tt{Last} instead of \tt{Step:} signalizes the end of data
\end{itemize}

Note that the term aggregate also refers to free molecules (i.e., unimers or
fully dissolved chains).

If \vtf format is used for the coordinates, the indices from \tt{agg} file can
be used in \href{http://www.ks.uiuc.edu/Research/vmd/}{vmd} to visualize, e.g.,
only a specific aggregate by using \tt{resid <id1> ... <id\#>} in the
\tt{Selected Atoms} box inside the vmd. A rough script to visualize aggregates
in different colours is in the \tt{Examples/scripts/Visualize} directory
(\tt{VisAgg*} files),
accompanied by the resultant picture (the bash script uses the
provided \vtf and \tt{agg} files to generate a \tt{tcl} script and run it via
\tt{vmd}).

An example of an \tt{agg} file can be found in the \tt{Examples/DistrAgg}
directory. %}}}

\section{Read system data} \label{sec:ReadSystemData}

\texttt{ReadStructure()} function reads all system information from
\texttt{vsf} and \texttt{vcf} files. \texttt{FIELD} file is used only to
get mass and charge of beads if the information is not in \texttt{vsf}
file. Provided \texttt{vsf} file is used
to get all information about beads and molecules -- names and numbers of
bead and molecules, bonds in molecules.  The first timestep of the
\texttt{vcf} file is used to determine numbers and ids of beads in that
\texttt{vcf} file which means that all timesteps must contain the same
beads.

The procedure in \texttt{ReadStructure()} is as follows:
\begin{enumerate}
  \item Go through \texttt{atom} section of the \texttt{vsf} file to
    identify default bead type (if \texttt{atom default} line is present),
    to find highest bead and molecule ids, and to find bead type names (and
    charges and masses if present).
  \item Go again through the \texttt{atom} section to read names and ids of
    beads and molecules as well as numbers of all beads and molecules for
    each type.
  \item Go through \texttt{bond} section of the \texttt{vsf} file to
    calculate number of bonds in each molecule type.
  \item Go again through the \texttt{bond} section to read bonds for each
    molecule type.
  \item Go through the \texttt{atom} section (for the third time) to assign
    bead ids to molecules.
  \item Go through the first timestep of \texttt{vcf} file to find which
    beads are in that \texttt{vcf} file (if no \texttt{vcf} file is
    provided -- e.g., for \texttt{DistrAgg} utility -- assume all beads are
    used).
  \item Modify all arrays to accommodate only the beads that are present in
    the \texttt{vcf} file.
  \item Read charge and mass from the \texttt{FIELD} file, if not already
    known from \texttt{vsf} file.
\end{enumerate}

\section{Commonly used structures} \label{sec:Struct}

% Box %{{{
\subsection[Box]{Simulation box size}\label{ssec:Box}
\ttb{typedef struct Box \{<members>\} BOX;}\\
\vspace{-1.75em}
\begin{longtable}{L{0.43\textwidth}p{0.595\textwidth}}
  \toprule
  member                         & explanation \\
  \midrule
  \ttb{(VECTOR)Length}           & simulation box length$^\star$\\
  \ttb{(VECTOR)TriLength}        & box length for triclinic box$^\dag$\\
  \ttb{(double)alpha}            & $\alpha$ in a triclinic cell (or 90$^\circ$
                                   for orthogonal cell)\\
  \ttb{(double)beta}             & $\beta$ in a triclinic cell (or 90$^\circ$
                                   for orthogonal cell)\\
  \ttb{(double)gamma}            & $\gamma$ in a triclinic cell (or 90$^\circ$
                                   for orthogonal cell)\\
  \ttb{(double [3][3])Transform} & transformation matrix to transform between
                                   \enquote{real} and \enquote{triclinic}
                                   coordinates$^\ddag$\\
  \ttb{(double [3][3])inverse}   & inverse of the transformation matrix\\
  \bottomrule
\end{longtable}
\note{$^\star$for triclinic cells, it corresponds to $a$, $b$, and $c$ (see
  \url{https://docs.lammps.org/Howto_triclinic.html})\\
$^\dag$$lx$, $ly$, and $lz$ in
  \url{https://docs.lammps.org/Howto_triclinic.html}
$^\ddag$see \url{https://www.ruppweb.org/Xray/tutorial/Coordinate system
transformation.htm}}
  %}}}

% Counts %{{{
\subsection[Counts]{Basic system information}\label{ssec:Counts}
\TODO update when I know how things are with file types other than \vtf\\
\ttb{typedef struct Counts \{<members>\} COUNTS;}\\
\vspace{-1.75em}
\begin{longtable}{L{0.36\textwidth}p{0.665\textwidth}}
  \toprule
  member           & explanation \\
  \midrule
  \ttb{(int)TypesOfBeads}           & number of bead types\\
  \ttb{(int)TypesOfMolecules}       & number of molecule types\\
  \ttb{(int)TypesOfBonds}$^\dag$    & number of bond types\\
  \ttb{(int)TypesOfAngles}$^\dag$   & number of Angle types\\
  \ttb{(int)TypesOfDiherals}$^\dag$ & number of Dihedral types\\
  \ttb{(int)Beads}$^*$              & total number of beads in a file with
                                      coordinates\\
  \ttb{(int)Bonded}$^*$             & total number of beads in all molecules\\
  \ttb{(int)Unbonded}$^*$           & total number of monomeric beads\\
  \ttb{(int)BeadsInVsf}$^*$         & total number of all beads in the system\\
  \ttb{(int)Molecules}$^*$          & total number of molecules\\
  \ttb{(int)Aggregates}             & total number of aggregates\\
  \bottomrule
\end{longtable}
\note{$^*$for \vtf, this number (usually) corresponds to a \vcf file, i.e., when
some beads defined in a \vsf file are missing from the \vcf file, these beads
are not (usually) included in this number\\
$^\dag$if the jparameters are unknown (such as for a \vsf
file), this has a value of \tt{-1}\TODO why not just say 0?}
  %}}}

% BeadType %{{{
\subsection[BeadType]{Information about bead types}\label{ssec:BeadType}
\ttb{typedef struct BeadType \{<members>\} BEADTYPE;} \\
\vspace{-1.75em}
\begin{longtable}{L{0.24\textwidth}p{0.785\textwidth}}
  \toprule
  member                & explanation\\
  \midrule
  \ttb{(char[17])Name}  & bead type name (at most 16 characters)\\
  \ttb{(int)Number}     & number of beads of the given type\\
  \ttb{(bool)Use}$^\P$  & \tt{true}/\tt{false} value whether these beads
                          should be used in a calculation\\
  \ttb{(bool)Write}$^\P$& \tt{true}/\tt{false} value whether these beads
                          should be written to an output coordinate file\\
  \ttb{(double)Charge}  & electric charge of the bead type (\tt{1000} if
                          undefined)\\
  \ttb{(double)Mass}    & mass of the bead type (\tt{0} if undefined)\\
  \ttb{(double)Radius}  & radius of the spherical bead (\tt{0} if undefined)\\
  \bottomrule
\end{longtable}
\note{$^\P$\TODO will be removed}
\begin{itemize}
  \item array size: \tt{Counts.TypesOfBeads}
\end{itemize} %}}}

% Bead %{{{
\subsection[Bead]{Information about individual beads}\label{ssec:Bead}
\ttb{typedef struct Bead \{<members>\} BEAD;} \\
\vspace{-1.75em}
\begin{longtable}{L{0.35\textwidth}p{0.675\textwidth}}
  \toprule
  member                               & explanation \\
  \midrule
  \ttb{(int)Type}                      & bead type index corresponding to a
                                         \ttb{struct BeadType} array index\\
  \ttb{(int)Molecule}                  & molecule index corresponding to a
                                         \ttb{struct Molecule} array index or
                                         $-1$ for an unbonded bead\\
  \ttb{(int)nAggregates}$^\P$          & number of aggregates the bead is in\\
  \ttb{(int *)Aggregate}$^\P$          & 1D array with aggregate indices\\
  \ttb{(int)Index}                     & index corresponding to an input file
                                         (e.g, \vsf file)\\
  \ttb{(struct Vector)Position}$^\dag$ & Cartesian coordinates\\
  \ttb{(struct Vector)Velocity}$^\dag$ & velocity of the bead\\
  \ttb{(bool) InTimestep}              & is the bead in the present timestep?\\
  \bottomrule
\end{longtable}
\note{$^\P$\TODO will be removed\\
$^\dag$ \ttb{struct Vector} contains members
\ttb{(double)x}, \ttb{(double)y}, and \ttb{(double)z}}
\begin{itemize}
  \item array size: \tt{Counts.Beads}
  \item array elements 0 to \tt{Counts.Unbonded} contain monomeric beads
  \item array elements \tt{Counts.Unbonded+1} to \tt{Counts.Beads} contain
    bonded beads
  \item \ttb{BEAD} is usually accompanied by an \ttb{(int *)Index} array (with
    size of \tt{Counts.BeadsInVsf}) connecting in-code bead indices with input
    file bead indices, i.e.,\\
    \ttb{Bead[<in-code index>].Index = <in-file index>} and\\
    \ttb{Index[<in-file index>] = <in-code index>}
    \begin{itemize}
      \item \TODO explain \ttb{Index} array usefulness
    \end{itemize}
\end{itemize} %}}}

% MoleculeType %{{{
\subsection[MoleculeType]{Information about molecule
  types}\label{ssec:MoleculeType}
\ttb{typedef struct MoleculeType \{<members>\} MOLECULETYPE;} \\
\vspace{-1.75em}
\begin{longtable}{L{0.27\textwidth}p{0.755\textwidth}}
  \toprule
  member                      & explanation \\
  \midrule
  \ttb{(char[9])Name}         & bead type name (at most 8 characters)\\
  \ttb{(int)Number}           & number of molecules of the given type\\
  \ttb{(int)nBeads}           & number of beads in these molecules\\
  \ttb{(int *)Bead}           & array with bead in-code indices (i.e.,
                                corresponding to \ttb{struct Bead} array
                                indices)\\
  \ttb{(int)nBonds}           & number of bonds in these molecules\\
  \ttb{(int **)Bond}$^\star$  & 2D array with two bead indices of the connected
                                beads and a bond type\\
  \ttb{(int)nAngles}          & number of angles in these molecules\\
  \ttb{(int **)Angle}$^\dag$  & 2D array with three bead indices of the beads in
                                the angle and an angle type\\
  \ttb{(int)nDihedrals}       & number of dihedral angles in these molecules\\
  \ttb{(int **)Dihedral}$^\S$ & 2D array with three bead indices of the beads in
                                the dihedral and a diheral type\\
  \ttb{(int)nBTypes}          & number of bead types in these molecules\\
  \ttb{(int *)BType}          & array with bead type indices (i.e.,
                                corresponding to a \ttb{struct BeadType} array
                                index)\\
  \ttb{(bool)InVcf}$^\P$      & \tt{true}/\tt{false} value whether these
                                molecules are present in the \vcf file\\
  \ttb{(bool)Use}$^\P$        & \tt{true}/\tt{false} value whether these
                                molecules should be used in a calculation\\
  \ttb{(bool)Write}$^\P$      & \tt{true}/\tt{false} value whether these
                                molecules should be written to an output
                                coordinate file\\
  \ttb{(double)Charge}$^\ddag$& total electric charge of the molecule type\\
  \ttb{(double)Mass}$^\ddag$  & total mass of the molecule type\\
  \bottomrule
\end{longtable}
\note{$^\star$size of \ttb{nBonds}\tt{$\times$3}: \ttb{Bond[i][0} and \ttb{1]}
  hold in-molecule bead indices (i.e., indices corresponding to
  \ttb{MoleculeType[].Bead} array) for bond $i$ and \ttb{Bond[i][2]} holds bond
  type (or \tt{-1} if bond type is undefined); bond types are defined in a
  \ttb{struct Params} array \\
$^\dag$size of \ttb{nAngles}\tt{$\times$4}: \ttb{Angle[i][0} to \ttb{2]} hold
  in-molecule bead indices (i.e., indices corresponding to
  \ttb{MoleculeType[].Bead} array) and \ttb{Angle[i][3]} holds angle type (or
  \tt{-1} if angle type is undefined); angle types are defined in a \ttb{struct
  Params} array\\
$^\S$size of \ttb{nDihedrals}\tt{$\times$5}: \ttb{Dihedral[i][0} to \ttb{3]}
  hold in-molecule bead indices (i.e., indices corresponding to
  \ttb{MoleculeType[].Bead} array) and \ttb{Dihedral[i][4]} holds dihedral type
  (or \tt{-1} if dihedral type is undefined); dihedral types are defined in a
  \ttb{struct Params} array\\
$^\P$\TODO will be removed\\
$^\ddag$undefined if any of the included bead types have undefined charge/mass}
\begin{itemize}
  \item array size: \tt{Counts.TypesOfMolecules}
\end{itemize} %}}}

% Molecule %{{{
\subsection[Molecule]{Information about individual
  molecules}\label{ssec:Molecule}
\ttb{typedef struct Molecule \{<members>\} MOLECULE;} \\
\vspace{-1.75em}
\begin{longtable}{L{0.27\textwidth}p{0.755\textwidth}}
  \toprule
  member                    & explanation\\
  \midrule
  \ttb{(int)Type}           & index of molecule type corresponding to a
                              \ttb{struct MoleculeType} array index\\
  \ttb{(int *)Bead}         & array with in-code bead indices (i.e.,
                              corresponding to \ttb{struct Bead} array indices);
                              size: \ttb{MoleculeType[].nBeads}\\
  \ttb{(int)Aggregate}$^\P$ & index of an aggregate the molecule is in
                              corresponding to a \ttb{struct Aggregate} array
                              index\\
  \ttb{(int)Index}          & index corresponding to an input file (e.g,
                              \tt{resid <id>} in \vsf file)\\
  \bottomrule
\end{longtable}
\note{$^\P$\TODO will be removed}
\begin{itemize}
  \item array size: \tt{Counts.Molecules}
\end{itemize} %}}}

% Aggregate %{{{
\subsection[Aggregate]{Information about individual
  aggregates}\label{ssec:Aggregates}
\TODO this will be changed (I guess)
\ttb{typedef struct Aggregate \{<members>\} AGGREGATE;} \\
\vspace{-1.75em}
\begin{longtable}{L{0.27\textwidth}p{0.755\textwidth}}
  \toprule
  member             & explanation \\
  \midrule
  \ttb{(int)nMolecules} & number of molecules in an aggregate\\
  \ttb{(int *)Molecule} & array with in-code molecule indices (i.e.,
                          corresponding to \ttb{struct Molecule} array indices\\
  \ttb{(int)nBeads}     & number of bonded beads in an aggregate\\
  \ttb{(int *)Bead}     & array with in-code indices of the bonded beads (i.e.,
                          corresponding to \ttb{struct Bead} array indices)\\
  \ttb{(int)nMonomers}  & number of monomeric beads in an aggregate\\
  \ttb{(int *)Monomer}  & array with in-code indices of the monomeric beads
                          (i.e., corresponding to \ttb{struct Bead} array
                          indices)\\
  \ttb{(double)Mass}    & total mass of an aggregate (undefined if any of
                          the molecules have undefined mass)\\
  \ttb{(bool)Use}$^\P$  & \tt{true}/\tt{false} value whether this aggregate
                          should be used in a calculation\\
  \bottomrule
\end{longtable}
\note{$^\P$\TODO will be removed}
\begin{itemize}
  \item array size: \tt{Counts.Aggregates}
\end{itemize} %}}}

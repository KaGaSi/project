\chapter{Introduction}

AnalysisTools is a set of utilities to (mainly) analyse trajectories of
particle-based simulations. The utilities (including one
from the \href{https://www.scd.stfc.ac.uk/Pages/DL_MESO.aspx}{DLMESO
simulation package}) can be roughly divided into four types:

\begin{itemize}
  \item utilities to calculate system-wide properties; e.g., pair
    correlation functions or particle densities along a simulation box axis
  \item utilities to calculate per-molecule or per-aggregate properties
    (where aggregate stands for any supramolecular structure); e.g., shape
    descriptors for individual molecules or whole micelles
  \item utilities to manipulate a configuration; e.g., create initial
    configuration for a simulation from scratch or by adding molecules to
    an existing one
  \item helper utilities to analyse text files; e.g., calculate averages
    and standard deviations of a data series
\end{itemize}

The \tt{Examples} directory contains examples showcasing capabilities of
individual utilities as well as possible workflows chaining several utilities.

\section{Installation}

AnalysisTools requires \tt{C} and \tt{Fortran} compilers and \tt{cmake}. The
utilities should be compiled in a separate directory, typically
\tt{<root>/build} (where \tt{<root>} is the AnalysisTools root directory). The
following generates a \tt{Makefile} within \tt{<root>/build}:

\tt{mkdir <root>/build; cd <root>/build; cmake ../}

To then compile the utilities, simply run \tt{make} (to compile all utilities)
or \tt{make <utility name>} (to compile a single utility) in \tt{<root>/build}.
The binaries are located in the \tt{<root>/build/bin} directory.

\section{DensityBox} \label{sec:DensityBox}

This utility calculates number density for all bead types along all three axis
directions of the simulation box, generating one file per axis. The density is
calculated from 0 to box length in the given direction, that is, the box is
\enquote{sliced} into blocks with width \tt{<width>}, and numbers of different
bead types are counted in each \enquote{slice}. Or to put it in other words, a
density profile for a bead type $i$ along an axis $\alpha$, $\rho(\alpha)$, is
calculated as
\begin{equation}
  \rho_i(\alpha) = \frac{\sum_j\delta(\alpha-\alpha_j^i)}
                   {\Delta\alpha L_\beta L_\gamma},
\end{equation}
where $\delta(\alpha-\alpha_i^j)$ gives the number of $i$ beads inside a slice
of the simulation box of the thickness $\Delta\alpha$ (i.e., \tt{<width>}) along
the $\alpha$-axis; $L_\beta$ and $L_\gamma$ represent box side lengths along the
two remaining axes.

The utility does not distinguish between beads with the same name in different
molecules, so if one bead type is in more than one molecule type, its density
will be averaged over all molecule types it appears in. If one requires
densities specific to certain molecules containing the same bead types, the
\tt{-x} option can be used to first run the utility without one molecule type
and then rerun it, excluding the other molecule type. Thus, two output files
(per axis) are generated, each missing densities from the specified molecule
types.

Note that this utility assumes orthogonal box with constant side lengths; in
case of triclinic box and/or varying box size, undefined behaviour may occur,
i.e., the utility may crash or freeze, and any results will not be reliable.

\vspace{1em}
\noindent
Usage: \tt{DensityBox <input> <width> <output> [options]}
\noindent
\begin{longtable}{p{0.24\textwidth}p{0.704\textwidth}}
  \toprule
  \multicolumn{2}{l}{Mandatory arguments}\\
  \midrule
  \tt{<input>} & input coordinate file (either \tt{vcf} or \tt{vtf} format)\\
  \tt{<width>} & width of each bin of the distribution\\
  \tt{<output>} & three output files with automatic \tt{-x.rho}, \tt{-y.rho},
    and \tt{-z.rho} endings\\
  \toprule
  \multicolumn{2}{l}{Options}\\
  \midrule
  \tt{-x <mol name(s)>} & exclude specified molecule type(s) (i.e., do not
    calculate density for beads in molecules \tt{<mol name(s)>})\\
      \midrule
  \multicolumn{2}{l}{Other options (see the beginning of 
                     Chapter~\ref{chap:Utils})}\\
  \midrule
  \multicolumn{2}{l}{\tt{-st},
                     \tt{-e},
                     \tt{-sk},
                     \tt{-i},
                     \tt{--help},
                     \tt{--verbose},
                     \tt{--silent},
                     \tt{--version}}\\
  \bottomrule
\end{longtable}

\noindent
Format of output files:
\begin{enumerate}[nosep,leftmargin=20pt]
  \item \tt{<output>} -- bead densities; one file per x-, y-, and z-axis
    \begin{itemize}[nosep,leftmargin=5pt]
      \item first line: AnalysisTools version
      \item second line: command used to generate the file
      \item third line: column headers
        \begin{itemize}[nosep,leftmargin=5pt]
          \item first is the centre of each bin (governed by
            \tt{<width>}); i.e., if \tt{<width>} is 0.1,
            then the centre of bin 0 to 0.1 is 0.05, centre of bin 0.1 to
            0.2 is 0.15, etc.
          \item the rest are for the calculated data: each column
            corresponds to the number density of the specified bead type
        \end{itemize}
    \item the rest of the file are data lines
    \end{itemize}
\end{enumerate}

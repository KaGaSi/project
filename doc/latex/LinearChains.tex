This section provides information about utilities with calculations that are sensible to do only on linear polymer chains. No check whether the molecules are linear is done.\hypertarget{LinearChains_EndToEnd}{}\subsection{End\+To\+End utility}\label{LinearChains_EndToEnd}
This utility calculates end-\/to-\/end distance of specified molecules. End-\/to-\/end distance makes sense only for linear chains, therefore it is assumed that the provided molecule names are linear chains. No check is performed. The distance is calculated between the first and the last bead of the molecule; that is, between the first and the last bead in the {\ttfamily F\+I\+E\+LD} entry for the given molecule. Also the use of joined coordinates (that is, without periodic boundary condition) is required, because the utility does not remove periodic boundary conditions.

The output is a file containing average end-\/to-\/end distance for every molecule type for each timestep.

Usage\+:

{\ttfamily End\+To\+End $<$input.\+vcf$>$ $<$output.\+vcf$>$ $<$molecule names$>$ $<$options$>$}

\begin{quote}
{\ttfamily $<$input.\+vcf$>$} \begin{quote}
input coordinate filename (must end with {\ttfamily .vcf}) containing either ordered or indexed timesteps (with joined coordinates) \end{quote}
{\ttfamily $<$output.\+vcf$>$} \begin{quote}
output filename with indexed coordinates (must end with {\ttfamily .vcf}) \end{quote}
{\ttfamily $<$molecule names$>$} \begin{quote}
names of molecule types (linear chains) to use \end{quote}
\end{quote}
\hypertarget{LinearChains_PersistenceLength}{}\subsection{Persistence\+Length utility}\label{LinearChains_PersistenceLength}
This utility calculates persistence length of specified molecules. It is assumed that the provided molecules are linear chains, but no check is performed. Also the use of joined coordinates (that is, without periodic boundary condition) is required, because the utility does not remove periodic boundary conditions.

The calculation of the persistence length is based on the projection of angles between bonds vectors (see e.\+g. \href{http://pubs.acs.org/doi/full/10.1021/ma012052u}{\tt this paper}).  
The following formula for the persistence length, $l_{\mathrm{t}}$ is used:

\begin{equation}
  l_{\mathrm{P}} = \langle b \rangle \sum_{i=0}^{i=N_b} \langle \cos
  \theta_i \rangle \mbox{,}
\end{equation}

where $\langle b \rangle$ is the average bond length in a molecule,
$\langle \theta_i \rangle$ is the average angle between two bond vectors
separated by $i$ bonds. $N_b$ is the number of bonds in the given molecule.


The output is a file containing average persistence length for every molecule type for each timestep.

Usage\+:

{\ttfamily Persistence\+Length $<$input.\+vcf$>$ $<$output.\+vcf$>$ $<$molecule names$>$ $<$options$>$}

\begin{quote}
{\ttfamily $<$input.\+vcf$>$} \begin{quote}
input coordinate filename (must end with {\ttfamily .vcf}) containing either ordered or indexed timesteps (with joined coordinates) \end{quote}
{\ttfamily $<$output.\+vcf$>$} \begin{quote}
output filename with indexed coordinates (must end with {\ttfamily .vcf}) \end{quote}
{\ttfamily $<$molecule names$>$} \begin{quote}
names of molecule types (linear chains) to use\end{quote}
\end{quote}
